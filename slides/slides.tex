\documentclass{beamer}
\usepackage[brazil]{babel}
\usepackage[utf8]{inputenc}

% Título e autores
\title{Implementação de Heurísticas em \emph{Prolog} para o Problema do Container}
\author{Bruno Barros Mello, Juliana Moura, Vitor Balestro}
\institute{Universidade Federal Fluminense (UFF)}
\date{\today}

\begin{document}

% Slide de título
\begin{frame}
	\titlepage
\end{frame}

% Table of contents
\begin{frame}{Conteúdo}
	\tableofcontents
\end{frame}

% Introdução
\section{Introdução e Apresentação do Problema}
\begin{frame}{Introdução e Apresentação do Problema}
	\begin{itemize}
		\item Objetivo: Maximizar o número de caixas empacotadas em um container.
		\item Desafios: Configurações de empacotamento em um espaço tri-dimensional.
		\item Heurísticas abordadas: Força Bruta, FFD, FFD com Rotações, LAFF.
	\end{itemize}
\end{frame}

% Força Bruta
\section{Força Bruta}
\begin{frame}{Força Bruta}
	\begin{itemize}
		\item Explora todas as combinações possíveis.
		\item Inviável para grandes instâncias devido ao alto custo computacional.
		\item Resultado: Estouro de limite de pilha ao em ambos conjuntos de entrada testados.
	\end{itemize}
\end{frame}

% FFD
\section{First Fit Decreasing}
\begin{frame}{First Fit Decreasing (FFD)}
	\begin{itemize}
		\item Ordena caixas por volume decrescente.
		\item Tenta encaixar cada caixa na primeira posição viável.
		\item Efetivo para maximizar o uso do espaço.
	\end{itemize}
\end{frame}

% FFD com Rotações
\section{FFD com Rotações}
\begin{frame}{FFD com Rotações}
	\begin{itemize}
		\item Similar ao FFD, mas considera rotações das caixas.
		\item Se uma caixa não cabe, tenta alguma variação de posição.
		\item Melhora o ajuste e eficiência em comparação ao FFD sem rotações.
		\item Resultados mais eficazes para grandes volumes de dados.
	\end{itemize}
\end{frame}

% LAFF
\section{Largest Area First Fit (LAFF)}
\begin{frame}{Largest Area First Fit}
	\begin{itemize}
		\item Ordena caixas pela área da base.
		\item Foca na ocupação de superfície.
		\item Menos eficiente em tempo em comparação ao FFD.
	\end{itemize}
\end{frame}

\begin{frame}{Resultados Obtidos (50x50x50)}
	\begin{table}[]
		\centering
		\resizebox{0.8\linewidth}{!}{%
			\begin{tabular}{lcc}
				\toprule
				\textbf{Método}        & \textbf{Caixas Empacotadas} & \textbf{Tempo de Execução} \\
				\midrule
				Força Bruta            & N/A                         & Estouro de Stack Limit     \\
				First Fit Decreasing   & 30                          & 0.338 s                    \\
				FFD com Rotações       & 30                          & 0.160 s                    \\
				Largest Area First Fit & 30                          & 21.911 s                   \\
				\bottomrule
			\end{tabular}
		}
		\caption{Container 50x50x50 com 30 caixas}
	\end{table}
\end{frame}

\begin{frame}{Resultados Obtidos (20x20x20)}
	\begin{table}[]
		\centering
		\resizebox{0.8\linewidth}{!}{%
			\begin{tabular}{lcc}
				\toprule
				\textbf{Método}        & \textbf{Caixas Empacotadas} & \textbf{Tempo de Execução} \\
				\midrule
				Força Bruta            & N/A                         & Estouro de Stack Limit     \\
				First Fit Decreasing   & 10                          & 0.121 s                    \\
				FFD com Rotações       & 10                          & 0.367 s                    \\
				Largest Area First Fit & 10                          & 0.425 s                    \\
				\bottomrule
			\end{tabular}
		}
		\caption{Container 20x20x20 com 15 caixas}
	\end{table}
\end{frame}

% Conclusões
\section{Conclusões}
\begin{frame}{Conclusões}
	\begin{itemize}
		\item Métodos baseados em volume são geralmente mais eficientes.
		\item Consideração de rotações melhora resultados.
		\item LAFF, apesar de atingir bons resultados para entradas menores, tem performance prejudicada exponencialmente para entradas maiores.
		\item Força bruta é impraticável exceto para pequenos casos.
	\end{itemize}
\end{frame}

\end{document}
